\documentclass{article}
\usepackage{fontspec}
\usepackage{xcolor}
\usepackage{sagetex}

\usepackage{euler}
\usepackage{amsmath}
\usepackage{amssymb}
\usepackage{unicode-math}


\usepackage[makeroom]{cancel}
\usepackage{ulem}

\setlength\parindent{0em}
\setlength\parskip{0.618em}
\usepackage[a4paper,lmargin=1in,rmargin=1in,tmargin=1in,bmargin=1in]{geometry}

\setmainfont[Mapping=tex-text]{Helvetica Neue LT Std 45 Light}

\newcommand\N{\mathbb{N}}
\newcommand\Z{\mathbb{Z}}
\newcommand\R{\mathbb{R}}
\newcommand\C{\mathbb{C}}
\newcommand\A{\mathbb{A}}

\usepackage{soul}

\begin{document}

\begin{center}
  165A --- Homework 2

  Ricardo J. Acuna

  (862079740)
\end{center}\vspace{1.618em}

\subsubsection*{Page 37}

\paragraph{1} For each of the functions below, describe the domain of
definition that is understood:

(a) $f(z) = \frac{1}{1-z^2};$

Rational functions are defined whenever the denominator is not zero.

$1-z^2 = 0 \implies z=\pm i$, so dom$(f) = \C\backslash \{\pm i\}.$

(b) $f(z) = \text{Arg}(\frac{1}{z});$

For the same reasons, dom$(f) = \C\backslash \{0\}.$

(c)$f(z) = \frac{z}{z-\bar{z}}$;

$z-\bar{z} = 0 \implies \frac{z-\bar{z}}{2} = \Re(z) = 0.$

However, $f(0)= 0$ and $\Re(0) = 0,$ so dom$(f) = \{z\in\C | z\neq 0
  \implies \Re(z) \neq 0\}$.

(d) $f(z) = \frac{1}{1-|z|^2}$;

$1-|z|^2 = 0 \implies |z| = 1, $so dom$(f) = \{z \in \C|\quad |z| \neq 1\}$.

\paragraph{3} Suppose that $f(z) = x^2-y^2 -2y +i(2x -2xy),$ where $z
= x+iy$. Use the expressions (see Sec. 5)
\[x = \frac{z+\bar{z}}{2} \text{, and } y = \frac{z-\bar{z}}{2i}\]

to write $f(z)$ in terms of $z$, and simplify the result.

\uwave{slu.}

$f(z) = x^2-y^2 -2y +i(2x -2xy) = x^2 -y^2 -2y +2ix -2ixy,$

$x^2 = \frac{z+\bar{z}}{2}^2 = \frac{z^2 +2z\bar{z} +\bar{z}^2}{4}$
and
$y^2 = \frac{z-\bar{z}}{2i}^2 = \frac{z^2 -2z\bar{z} +\bar{z}^2}{-4} = \frac{-z^2 +2z\bar{z} -\bar{z}^2}{4}
$
$\implies x^2 - y^2 = \frac{z^2 +2z\bar{z} +\bar{z}^2 -(-z^2
  +2z\bar{z} -\bar{z}^2)}{4}$

$\implies x^2 -y^2 = \frac{z^2 +\bar{z}^2}{2},$

$-2y = -2\frac{z-\bar{z}}{2i} = \frac{-z+\bar{z}}{i}$ and $(\frac{i}{i}
= 1 = i(-i) \implies \frac{1}{i} = -i)$

$\implies -2y = -i(-z+\bar{z}) = iz-i\bar{z},$

$2ix = 2i\frac{z+\bar{z}}{2} = iz+i\bar{z}$
$ \implies 2ixy =
(iz+i\bar{z})(\frac{z-\bar{z}}{2i}) = \frac{z^2-z\bar{z} +z\bar{z}
  -\bar{z}^2}{2} = \frac{z^2 -\bar{z}^2}{2}$

$\implies 2ix -2ixy = iz+i\bar{z} -\frac{z^2 -\bar{z}^2}{2}.$

$\implies f(z) = \frac{z^2 +\bar{z}^2}{2} +iz -i\bar{z} + iz+i\bar{z}
-\frac{z^2 -\bar{z}^2}{2}. $

$\implies f(z) = \bar{z}^2 +2iz\quad $ $\lozenge$
\newpage
\paragraph{4} Write the function

\[f(z) = z + \frac{1}{z} \quad (z\neq 0)\]

in the form $f(z) = u(r,\theta) +iv(r,\theta).$

\uwave{slu.}

Let $z = r(\cos{\theta} +i\sin{\theta}),$

$\implies \frac{1}{z} = \frac{1}{r(\cos{\theta}
  +i\sin{\theta})}\frac{\cos{\theta} -i\sin{\theta}}{\cos{\theta}
  -i\sin{\theta}} =  \frac{\cos{\theta}
  -i\sin{\theta}}{r(\cos^2{\theta} +\sin^2{\theta})} =\frac{\cos{\theta}
  -i\sin{\theta}}{r}$

$\implies z +\frac{1}{z} = r(\cos{\theta} +i\sin{\theta}) + \frac{\cos{\theta}
  -i\sin{\theta}}{r} =
(r+\frac{1}{r})\cos{\theta}+i(r-\frac{1}{r})\sin{\theta}$$\lozenge$

\subsubsection*{Page 44}

\paragraph{3} Sketch the region onto which the sector $r\leq 1, 0\leq
\theta \leq \pi/4$ is mapped by the transformation (a) $w = z^2$;(b)
$w= z^3$; (c) $w = z^4$.

\uwave{slu.}
\begin{sagesilent}
  from sage.plot.disk import disk
  s = disk((0.0,0.0), 1, (0, pi/4), color='teal')
  a = disk((0.0,0.0), 1, (0, pi/2), color='yellow')
  b = disk((0.0,0.0), 1, (0, 3*pi/4), color='orange')
  c = disk((0.0,0.0), 1, (0, pi), color='red')
\end{sagesilent}

$\sageplot[scale = 0.5]{s}$

under $z^2$ the region maps to $\sageplot[scale = 0.5]{a}$

under $z^3$ the region maps to $\sageplot[scale = 0.5]{b}$

under $z^4$ the region maps to $\sageplot[scale = 0.5]{c}$

\paragraph{4} Show that the lines $ay = x (a \neq 0)$ are mapped onto the spirals $ρ = \exp(aφ)$ under
the transformation $w = \exp z$, where  $w = ρ \exp(iφ)$.

\uwave{slu.}

\begin{sagesilent}
  from sage.plot.line import line
  l1 = line([(-1.1,-1.1), (1.1,1.1)])
  l2 = line([(-1.1/2,-1.1), (1.1/2,1.1)], color='green')
  l3 = line([(-1.1*-2,-1.1), (1.1*-2,1.1)], color='orange')
  from sage.plot.plot import parametric_plot
  def spiral (a, c):
      t = var('t')
      x = exp(abs(t)*sqrt(a^2+1)*cos(atan(1/a)))*cos(abs(t)*sqrt(a^2+1)*sin(atan(1/a)))
      y = exp(abs(t)*sqrt(a^2+1)*cos(atan(1/a)))*sin(abs(t)*sqrt(a^2+1)*sin(atan(1/a)))
      return parametric_plot((x,y),(t,-20,20),color = c)

  s1 = spiral(1,'blue')
  s2 = spiral(1/2,'green')
  s3 = spiral(-2,'orange')
\end{sagesilent}


  Let $z = x +iy$.

  $ay = x \implies z = ay + iy$

  $\implies |z| = \sqrt{a^2y^2+y^2} = |y|\sqrt{a^2+1}$.

  $\implies \gamma := \sqrt{a^2 + 1}$ is fixed.

  and

  $x = ay $

  $\implies$

  $ \phi := \text{atan}{(y/ay)} = \text{atan}{(1/a)}.$

  $\implies \phi$ is fixed.

  $\implies z = |y|\gamma\cos(\phi) +i |y|\gamma\sin(\phi)$




$w = \exp (z) = \exp(|y|\gamma\cos(\phi) +i |y|\gamma\sin(\phi)) =$

$
\exp(|y|\gamma\cos(\phi))\exp(i|y|\gamma\sin(\phi)) =$

$
  \exp(|y|\gamma\cos(\phi))(\cos(|y|\gamma\sin(\phi))
  +i\sin(|y|\gamma\sin(\phi)))=$

  $
  \exp(|y|\sqrt{a^2 + 1}\cos(\text{atan}{(1/a)}))(\cos(|y|\sqrt{a^2 + 1}\sin(\text{atan}{(1/a)})) +i\sin(|y|\sqrt{a^2+1}\sin(\text{atan}{(1/a)})))$

\begin{minipage}{0.39\textwidth}
$\sageplot[scale = 0.38]{l1+l2+l3}$
\end{minipage}
\begin{minipage}{0.39\textwidth}
$\sageplot[scale = 0.38]{s1}$
\end{minipage}

\begin{minipage}{0.39\textwidth}
$\sageplot[scale = 0.38]{s2}$
\end{minipage}
\begin{minipage}{0.39\textwidth}
$\sageplot[scale = 0.38]{s3}$
\end{minipage}

They are spirals because as $|y|\rightarrow 0$, $\exp(A|y|)
\rightarrow 1,\forall A$ constants. But, it does go through all of the
angles, because cos and sin are cyclic over the positive reals.

\paragraph{7} Find the image of the semi-infinite strip  $x\geq 0$,
$0\leq y < \pi$, under the trasformation $w = \exp z$.



\begin{sagesilent}
  from sage.plot.contour_plot import region_plot
  x,y = var('x,y')
  s7 = region_plot((x>= 0, 0<=y,y<pi),(x,-1.0,20.0),(y,-1.0,3.2))
  s8 = region_plot((y>0, x^2+y^2>=1), (x,-5.0,5.0),(y,0.0,1.2))

  from sage.plot.plot import parametric_plot
  t = var('t')
  e7 = parametric_plot((exp(0)*cos(t),exp(0)*sin(t)), (t, 0,pi))
  e8 = parametric_plot((exp(1/2)*cos(t),exp(1/2)*sin(t)), (t, 0,pi))
  e9 = parametric_plot((exp(1/4)*cos(t),exp(1/4)*sin(t)), (t, 0,pi))
  e10 = parametric_plot((exp(10)*cos(t),exp(10)*sin(t)), (t, 0,pi))
\end{sagesilent}
$\sageplot{s7}$

Let $z = x+iy \implies w = \exp(z) = \exp(x+iy) = \exp(x)\exp(iy)$

$x \geq 0 \implies$ the radius $\rho = \exp(x)$ of $w$ is increasing,
and $\rho \geq 1$.

$0\leq y <\pi \implies$ the angle of $w$ runs from $0$ to $\pi$, not
including $\pi.$

We can compute the following to see the pattern.

$\sageplot[scale = 0.1]{e7}$:  $\rho = 1$,
$\sageplot[scale = 0.1]{e10}$: $\rho = \exp(10)$,
$\sageplot[scale = 0.1]{e8}$: $\rho = \exp(1/2)$,
$\sageplot[scale = 0.1]{e9}$: $\rho = \exp(1/4)$.

So, the map wraps the semi-infinite strip into the upper half-plane,
minus the open unit circle. And not including the ray that starts from
$-1+0i$ along the $x-$axis to $-\infty + 0i.$

$\sageplot[scale = 0.5]{s8}$
\newpage
\paragraph{8} One interpretation of a function $w = f(z) = u(x,y) + iv(x,y)$ is that of a vector field
in the domain of definition of $f$. The function assigns a vector $w$, with components
$u(x, y)$ and $v(x, y)$, to each point $z$ at which it is defined. Indicate graphically the
vector fields represented by (a) $w = iz$; (b) $w = z/|z|$.

(a) First we want to express $w = u +iv$.

$z = x+iy $ and $  w = iz \implies w = ix +i^2y = -y + ix \implies u =
-y$ and $v = x$.

\begin{sagesilent}
  x,y = var('x,y')
  v1 = sage.plot.plot_field.plot_vector_field((-y,x),
  (x,-5,5),(y,-5,5))
  v2 = sage.plot.plot_field.plot_vector_field((x/(x^2+y^2),y/(x^2+y^2)), (x,-5,5),(y,-5,5))
\end{sagesilent}

$\sageplot[scale = 0.75]{v1}$

(b) First we want to express $w = u +iv$.

$z = x+iy $ and $  w = z/|z| \implies w = (x +iy)/(x^2+y^2) = \frac{x}{x^2+y^2} + i\frac{y}{x^2+y^2} \implies u =
\frac{x}{x^2+y^2}$ and $v = \frac{y}{x^2+y^2}$.
$\sageplot[scale = 0.75]{v2}$
\newpage

\subsubsection*{Page 55}
\paragraph{1} Use definition (2), Sec. 15, of limit to prove that

(a) $\lim_{z\rightarrow z_0} \Re(z) = \Re(z_0)$;

\uwave{slu.}

WTS $\forall \epsilon > 0: \exists \delta > 0: |z-z_0|<\delta \implies
|\Re(z) - \Re(z_0)| < \epsilon$

$|\Re(z) - \Re(z_0)| < \epsilon \iff |\frac{z+\bar{z}}{2} - \frac{z_0
  \bar{z_0}}{2}| < \epsilon \iff |z + \bar{z} - z_0
  - \bar{z_0}|/2 <\epsilon \iff |z - z_0 +\bar{z} -
  \bar{z_0}| < 2\epsilon$
  $\iff  |z-z_0 +\bar{z}-\bar{z_0}| < |z-z_0| + |\bar{z} -
  \bar{z_0}| = \epsilon.$

  Now, $|\bar{z} -\bar{z_0}| = |\bar{z -z_0}| = |z-z_0|,$ so all of
  the above is if and only if, $|\Re(z)-\Re(z_0)|< 2|z-z_0| < 2\delta
  = 2\epsilon$.

  Put $\delta = \epsilon$.
  So, that shows it. $\square$

  (b) $\lim_{z\rightarrow z_0} \bar{z} = \bar{z_0}$;

  \uwave{slu.}

  WTS $\forall \epsilon > 0:\exists \delta > 0: |z-z_0|< \delta
  \implies |\bar{z} - \bar{z_0}|<\epsilon$.

  Put $z = x + i y$, and $z_0 = x_0 + i y_0$

  $|\bar{z} - \bar{z_0}|<\epsilon \iff |x-iy - (x_0 -iy_0)|<\epsilon
  $

  $\iff |x-x_0 -iy+iy_0|<|x-x_0|+|i(-y+y_0)| = |x-x_0|+|i||y_0-y| =
  |x-x_0| +|y-y_0|= \epsilon$.

$|z-z_0|= |x+iy-x_0-iy_0| < |x-x_0|+|i(y-y_0)| =
|x-x_0|+|y-y_0|<\delta$

Put $\delta = \epsilon$, and we are done!
  $\square$

(c) $\lim_{z\rightarrow 0} \frac{\bar{z}^2}{z} = 0$;

\uwave{slu.}

WTS $\forall \epsilon > 0:\exists \delta > 0: |z-0|< \delta
\implies |\frac{\bar{z}^2}{z} - 0|<\epsilon$.

$|\frac{\bar{z}^2}{z} - 0|<\epsilon \iff |\frac{\bar{z}^2}{z}| =
\frac{|\bar{z}^2|}{|z|} =\frac{|\bar{z^2}|}{|z|} = \frac{|z^2|}{|z|} =
\frac{|z|^2}{|z|} = |z|<\epsilon$

So, put $\delta = \epsilon\quad \square$

\paragraph{3}. Let $n$ be a positive integer and let $P(z)$ and $Q(z)$ be polynomials, where $Q(z_0) \neq 0$.
Use Theorem 2 in Sec. 16, as well as limits appearing in that section,
to find

(a) $\lim_{z\rightarrow z_0 \frac{1}{z^n}}\quad (z_0\neq 0)$;

\uwave{slu.}
$\lim_{z\rightarrow z_0} 1 = 1,$ and $\lim_{z\rightarrow z_0} z^n =
z_0^n$.

Given $z_0\neq 0 \implies z_0^n \neq 0$ for all positive integers.

By theorem 2.(10) $\lim_{z\rightarrow z_0} \frac{1}{z^n} =
\frac{1}{z_0^n}\quad \lozenge$

(b) $\lim_{z\rightarrow i \frac{iz^3 - 1}{z + i}}$;

\uwave{slu.}

$\lim_{z\rightarrow i} i = i$  and $\lim_{z\rightarrow i} z^3 = i^3$

By theorem 2.(9)  $\lim_{z\rightarrow i} iz^3 = i*i^3 = 1$

$\lim_{z\rightarrow i} -1 = -1$, so by theorem 2.(8)
$\lim_{z\rightarrow i} iz^3-1 = 1-1 = 0$.

$\lim_{z\rightarrow i} z = i$, so by theorem 2.(8) $\lim_{z\rightarrow
  i} z+i = i+i = 2i \neq 0$.

So, by theorem 2.(10) $\lim_{z\rightarrow i} \frac{iz^3 - 1}{z + i} =
0 \quad \lozenge$

(c) $\lim_{z\rightarrow z_0} \frac{P(z)}{Q(z)} =
\frac{P(z_0)}{Q(z_0)},$ by repeated applications of theorem 2, we get
limits of polynomials are evaluations, and
observing $Q(z_0)\neq 0$ we see the limit exists and is that $\lozenge$

\paragraph{5} Show that the limit of the function
\[f(z) = (\frac{z}{\bar{z}})^2\]

as $z$ tends to $0$ does not exist. Do this by letting non-zero points
$z = (x,0)$, and $z = (x,x)$ approach the origin. [Note that it is not sufficient to simply consider points
$z = (x, 0)$ and $z = (0, y)$, as it was in Example 2, Sec. 15.]

\uwave{slu.}

Let $z = x +0i \implies f(z) = (\frac{x}{x})^2 = 1$, so as
$x\rightarrow 0,$ $f(x+0i)\rightarrow 1.$

Let $z = x+ix \implies f(z) = (\frac{x+ix}{x-ix})^2=
(\frac{x(1+i)}{x(1-i)})^2= (\frac{1+i}{1-i})^2 =
(\frac{1+i}{1-i}\frac{1+i}{1+i})^2 = (\frac{(1+i)^2}{1+1})^2 =
(\frac{2i}{2})^2 = -1$.
, so as
$x+ix\rightarrow 0,$ $f(x+ix)\rightarrow -1.$

If the limit exists it is unique, therefore the limit doesn't exist
$\square$
\paragraph{7} 7. Use definition (2), Sec. 15, of limit to prove that
if $\lim_{z\rightarrow z_0} f(z) = w_0$, then $\lim_{z\rightarrow z_0}
|f(z)| = |w_0|$

Suggestion: Observe how the first of inequalities (9), Sec. 4, enables
one to write
$||f(z)| -|w_0||\leq |f(z)-w_0|$

\uwave{pf.}

$\lim_{z\rightarrow z_0} f(z) = w_0$ means that,

$\forall \epsilon>0:\exists \delta >0:  |z-z_0|< \delta \implies
|f(z)-w_0|<\epsilon$

by the reverse triangle inequality,

$||f(z)| -|w_0||\leq |f(z)-w_0| <\epsilon$, so we are done, because
the same $\delta$ works
$\blacksquare$

\paragraph{10}  Use the theorem in Sec. 17, to show that

(a) $\lim_{z\rightarrow \infty} \frac{4z^2}{(1-z)^2} = 4$;

\uwave{slu.}

$f(z) = \frac{4z^2}{(1-z)^2} =\frac{4z^2}{z^2 -2z +1}\frac{1/z^2}{1/z^2} =
\frac{4}{1 -\frac{2}{z} +\frac{1}{z^2}}$

$\implies f(1/z) = \frac{4}{1 -\frac{2}{1/z} +\frac{1}{(1/z)^2}} =
\frac{4}{1 -2z +z^2}$

$\lim_{z\rightarrow 0} f(1/z) = 4$, so by the theorem
$\lim_{z\rightarrow \infty} f(z) = 4\quad \lozenge$

(b) $\lim_{z\rightarrow 1} \frac{1}{(z-1)^3} = \infty;$

\uwave{slu.}

Consider, $\lim_{z\rightarrow 1} \frac{1}{\frac{1}{(z-1)^3}} = (z-1)^3
= 0 \implies \lim_{z\rightarrow 1} \frac{1}{(z-1)^3} = \infty$ by the
theorem in Sec. 17$\quad \lozenge$

(c) $\lim_{z\rightarrow \infty} \frac{z^2+1}{z-1} = \infty;$

\uwave{slu.}
$f(z)= \frac{z^2+1}{z-1} \implies f(1/z) = \frac{(1/z)^2+1}{(1/z) -1}
\implies 1/f(1/z) = \frac{(1/z) -1}{1/z^2 +1} =
\frac{\frac{1-z}{z}}{\frac{1+z^2}{z^2}} =
\frac{\frac{1-z}{z}}{\frac{1+z^2}{z^2}} = \frac{(1-z)z^2}{(1+z^2)z} =
\frac{z-z^2}{1+z^2}$

$\lim_{z\rightarrow 0} \frac{z-z^2}{1+z^2} = 0 \implies
\lim_{z\rightarrow \infty} \frac{z^2+1}{z-1} = \infty \quad \lozenge$

\newpage
\paragraph{11} With the aid of theorem in Sec. 17, show that when

\[T(z) = \frac{az+b}{cz+d}\quad ad-bc \neq 0,\]

(a) $\lim_{z\rightarrow \infty} T(z) = \infty$ if $c=0$;

\uwave{slu.}

$c = 0 \implies T(z) = \frac{az+b}{d} = \frac{a}{d} z +\frac{b}{d}$

$\implies T(1/z) = \frac{a}{dz} +\frac{b}{d} = \frac{a+bz}{dz}$

$\implies 1/T(1/z) = \frac{dz}{a+bz}$

$\lim_{z\rightarrow 0} \frac{dz}{a+bz} = 0 \implies \lim_{z\rightarrow
  \infty} T(z) = \infty \quad \lozenge$

(b) $\lim_{z\rightarrow \infty} T(z) = \frac{a}{c}$, and
$\lim_{z\rightarrow -d/c} T(z) = \infty$ if $c\neq 0$;

\uwave{slu.}

$c\neq 0 \implies T(z) = \frac{az+b}{cz+d}$

$\implies T(1/z) = \frac{a(1/z)+b}{c(1/z)+d}
=\frac{\frac{a+bz}{z}}{\frac{c+dz}{z}} = \frac{a+bz}{c+dz}$

$\lim_{z\rightarrow 0} T(1/z) = \frac{a}{c} \implies
\lim_{z\rightarrow \infty} T(z) = \frac{a}{c}.$

$\implies 1/T(z) = \frac{cz+d}{az+b}$

$\implies \lim_{z\rightarrow -d/c} 1/T(z) =
\frac{c(-d/c)+d}{a(-d/c)+b} = 0,$
which works because $ad-bc \neq 0 \implies ad \neq bc.$

$\implies \lim_{z\rightarrow -d/c} T(z) = \infty \quad \lozenge$
\end{document}

%%% Local Variables:
%%% mode: latex
%%% TeX-master: t
%%% End:
