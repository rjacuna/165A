\documentclass{article}
\usepackage{fontspec}
\usepackage{xcolor}
\usepackage{sagetex}

\usepackage{euler}
\usepackage{amsmath}
\usepackage{amssymb}
\usepackage{unicode-math}


\usepackage[makeroom]{cancel}
\usepackage{ulem}

\setlength\parindent{0em}
\setlength\parskip{0.618em}
\usepackage[a4paper,lmargin=1in,rmargin=1in,tmargin=1in,bmargin=1in]{geometry}

\setmainfont[Mapping=tex-text]{Helvetica Neue LT Std 45 Light}

\newcommand\N{\mathbb{N}}
\newcommand\Z{\mathbb{Z}}
\newcommand\R{\mathbb{R}}
\newcommand\C{\mathbb{C}}
\newcommand\A{\mathbb{A}}

\usepackage{soul}

\begin{document}

\begin{center}
  165A --- Homework 7

  Ricardo J. Acuna

  (862079740)
\end{center}\vspace{1.618em}

\subsubsection*{Page 170}

\paragraph{3} Let $C$ be the circle $|z| = 3$, described in the
positive sense. Show that if$$g(z) = \int_{C} \frac{2s^2 -s -2}{s-z}
ds\quad (|z|\neq 3),$$ then $g(2) = 8\pi i.$ What is the value of
$g(z)$ when $|z|>3$ ?

\uwave{slu. }

$|2| = 2 < 3 \implies 2 \in \text{int}(C) \implies g(2) = 2\pi i
(2(2)^2-2-2) = 2\pi i (8-4) = 8\pi i$

$\forall z\in \C: |z|\neq 3 \implies s-z \neq 0 \implies \frac{2s^2 -s
  -2}{s-z}$ is analytic on $\C \backslash \{z\in \C||z|\neq 3\}$

$C$ is a closed curve and $C \subset \C\backslash \{z\in \C|\quad
|z|\neq 3\}\quad \blacklozenge$


\paragraph{10} Let $f$ be an entire function such that $|f(z)|\leq A
|z|$ for all $z$, when $A$ is a fixed positive number. Show that $f(z)
= a_1 z$, where $a_1$ is a complex constant.

\uwave{slu. }

$$|f^{(n)}(z)| \leq \frac{n!M_R}{R^n} \implies |f^{(2)}(z)| \leq
\frac{2M_R}{R^2} \leq \frac{2A(|z|+R)}{R^2} \forall z\in \C$$

Since $f$ is entire, then the radius $R$ where it is analytic is
arbitrarily large. So let $R\rightarrow \infty$, gives the right hand
side of the last inequality is zero.

$$|f^{(2)}(z_0)| \leq 0 \implies f^{(2)}(z) = 0\quad \forall z\in \C
\implies \exists a_1,a_2 \in \C : f(z) = a_1z +a_2$$

But, $|f|\leq A|z|\quad \forall z\in \C \implies a_2 = 0 \implies f(z) =
a_1z \quad \blacklozenge$



\subsubsection*{Page 178}

\paragraph{3} Let a function $f$ be continuous on a closed bounded
region $R$, and let it be analytic and not constant throughout the
interior of $R$. Assuming that $f(z)\neq 0$ anywhere in $R$, prove
that $|f(z)|$ has a\textit{ minimum value m} in $R$ which occurs on
the boundary of $R$ and never in the interior. Do this by applying the
corresponding result for maximum values (Sec. 54) to the function
$g(z) = 1/f(z).$

\uwave{pf.}

Note, that $\forall z\in R, f(z)\neq 0 \implies g(z) = 1/f(z)$ is
well-defined analytic and non-constant throughout the interior of
$R$. Then the maximum of $|g(z)|$ is always reached and it is located
somewhere in $\partial R$, by the corollary in (Sec. 54).


Let $m = \min |f(z)| \implies \forall z\in R,  m \leq |f(z)| \implies
\forall z \in R, |g(z)| = \frac{1}{|f(z)|} \leq \frac{1}{m} = \max
|g(z)|\quad \blacksquare$


\end{document}

%%% Local Variables:
%%% mode: latex
%%% TeX-master: t
%%% End:
